% CS 111 style
% Typical usage (all UPPERCASE items are optional):
%       \input 111pre
%       \begin{document}
%       \MYTITLE{Title of document, e.g., Lab 1\\Due ...}
%       \MYHEADERS{short title}{other running head, e.g., due date}
%       \PURPOSE{Description of purpose}
%       \SUMMARY{Very short overview of assignment}
%       \DETAILS{Detailed description}
%         \SUBHEAD{if needed} ...
%         \SUBHEAD{if needed} ...
%          ...
%       \HANDIN{What to hand in and how}
%       \begin{checklist}
%       \item ...
%       \end{checklist}
% There is no need to include a "\documentstyle."
% However, there should be an "\end{document}."
%
%===========================================================
\documentclass[11pt,twoside,titlepage]{article}
%%NEED TO ADD epsf!!
\usepackage{threeparttop}
\usepackage{graphicx}
\usepackage{latexsym}
\usepackage{color}
\usepackage{listings}
\usepackage{fancyvrb}
%\usepackage{pgf,pgfarrows,pgfnodes,pgfautomata,pgfheaps,pgfshade}
\usepackage{tikz}
\usepackage[normalem]{ulem}
\tikzset{
    %Define standard arrow tip
%    >=stealth',
    %Define style for boxes
    oval/.style={
           rectangle,
           rounded corners,
           draw=black, very thick,
           text width=6.5em,
           minimum height=2em,
           text centered},
    % Define arrow style
    arr/.style={
           ->,
           thick,
           shorten <=2pt,
           shorten >=2pt,}
}
\usepackage[noend]{algorithmic}
\usepackage[noend]{algorithm}
\newcommand{\bfor}{{\bf for\ }}
\newcommand{\bthen}{{\bf then\ }}
\newcommand{\bwhile}{{\bf while\ }}
\newcommand{\btrue}{{\bf true\ }}
\newcommand{\bfalse}{{\bf false\ }}
\newcommand{\bto}{{\bf to\ }}
\newcommand{\bdo}{{\bf do\ }}
\newcommand{\bif}{{\bf if\ }}
\newcommand{\belse}{{\bf else\ }}
\newcommand{\band}{{\bf and\ }}
\newcommand{\breturn}{{\bf return\ }}
\newcommand{\mod}{{\rm mod}}
\renewcommand{\algorithmiccomment}[1]{$\rhd$ #1}
\newenvironment{checklist}{\par\noindent\hspace{-.25in}{\bf Checklist:}\renewcommand{\labelitemi}{$\Box$}%
\begin{itemize}}{\end{itemize}}
\pagestyle{threepartheadings}
\usepackage{url}
\usepackage{wrapfig}
% \usepackage{hyperref}
\usepackage[hidelinks]{hyperref}
%=========================
% One-inch margins everywhere
%=========================
\setlength{\topmargin}{0in}
\setlength{\textheight}{8.5in}
\setlength{\oddsidemargin}{0in}
\setlength{\evensidemargin}{0in}
\setlength{\textwidth}{6.5in}
%===============================
%===============================
% Macro for document title:
%===============================
\newcommand{\MYTITLE}[1]%
   {\begin{center}
     \begin{center}
     \bf
     CMPSC 111\\Introduction to Computer Science I\\
     Fall 2014\\
     \medskip
     \end{center}
     \bf
     #1
     \end{center}
}
%================================
% Macro for headings:
%================================
\newcommand{\MYHEADERS}[2]%
   {\lhead{#1}
    \rhead{#2}
    \immediate\write16{}
    \immediate\write16{DATE OF HANDOUT?}
    \read16 to \dateofhandout
    \lfoot{\sc Handed out on \dateofhandout}
    \immediate\write16{}
    \immediate\write16{HANDOUT NUMBER?}
    \read16 to\handoutnum
    \rfoot{Handout \handoutnum}
   }

%================================
% Macro for bold italic:
%================================
\newcommand{\bit}[1]{{\textit{\textbf{#1}}}}

%=========================
% Non-zero paragraph skips.
%=========================
\setlength{\parskip}{1ex}

%=========================
% Create various environments:
%=========================
\newcommand{\PURPOSE}{\par\noindent\hspace{-.25in}{\bf Purpose:\ }}
\newcommand{\SUMMARY}{\par\noindent\hspace{-.25in}{\bf Summary:\ }}
\newcommand{\DETAILS}{\par\noindent\hspace{-.25in}{\bf Details:\ }}
\newcommand{\HANDIN}{\par\noindent\hspace{-.25in}{\bf Hand in:\ }}
\newcommand{\SUBHEAD}[1]{\bigskip\par\noindent\hspace{-.1in}{\sc #1}\\}
%\newenvironment{CHECKLIST}{\begin{itemize}}{\end{itemize}}

\begin{document}
\MYTITLE{Lab 3 \\9 September 2015\\Due Wednesday, 16 September by 2:30 pm}

\vspace{-0.275in}
\subsection*{Objectives}

To gain more experience working with variables and expressions you will write a Java program that performs user input by
correctly employing a {\tt java.util.Scanner} object and its methods.

\vspace{-0.15in}
\subsection*{General Guidelines for Labs}
\begin{itemize}
\itemsep -.25pt
\item
{\bf Work on the Alden Hall computers.} If you want to work on a different
machine, be sure to transfer your programs to the Alden
machines and re-run them before submitting.
\item
{\bf Update your repository often!} You should add, commit,
and push your updated files each time you work on them.  I will not grade
your programs until the due date has passed.
\item
{\bf Review the Honor Code policy.} You
may discuss programs with others, but programs that are nearly identical
to others will be taken as evidence of violating the Honor Code.
\end{itemize}

\vspace{-0.25in}
\subsection*{Reading Assignment}

To learn more about variables, expressions, and user input, review Sections 2.1--2.6 in your textbook. Please pay
close attention to the {\tt Scanner} methods in Figure 2.7 and the program in Listing 2.8.

\vspace{-0.15in}
\subsection*{Create a New Directory and Read the Template}

In your {\tt cs111F2014-<your user name>} directory type the command ``{\tt mkdir lab3}'' to create a new directory for
the third laboratory.  \noindent Type ``{\tt cd lab3}'' to change into this new directory.  To create the required file,
type ``{\tt gvim Lab3.java}''. Begin your program by including the {\tt Template.java} file that you created during
the last laboratory session (if you don't have one, you can still create one if you'd like---it will save you time!).
Assuming that your {\tt Template.java} file is inside the {\tt labs/} directory, but not inside the {\tt lab3/}
directory, you need to type: ``{\tt :r../Template.java}'' in {\tt gvim} to read your program template.  See last week's
laboratory assignment for more information about creating and using the template. Remember, in {\tt gvim} and the
terminal window ``{\tt ..}'' stands for ``go back one directory'' and ``{\tt .}'' means ``the current directory''.

\vspace{-0.15in}
\subsection*{Tip and Bill Calculator}

This laboratory assignment asks you to write a Java program named ``{\tt Lab3.java}'' that will calculate the tip, the
total bill for the user, and each person's share of the restaurant bill (if there are two or more people). In
particular, your program needs to do the following:

\vspace*{-.1in}
\begin{enumerate}

\item Ask the user to enter a name (remember to save the user's input into a variable).

\item Using this name, display a friendly and appropriate welcome message to the user.

\item Ask the user to enter the restaurant's bill amount (remember to save the user's input into an appropriate
  variable). You should allow for floating point (e.g., decimal or fractional) values.

\item Ask the user to enter the desired tip percentage as a number between $0$ and $100$ or as a decimal number between
  $0$ and $1$. You have to decide which range you want to use for your program and specify it when prompting for the
  user's input (after picking the correct data type for the percentage, remember to save the user's input into an
  appropriate variable).

\item Calculate the tip as $tip = \frac{percentage}{100} \times bill$ if percentage ranges between $0$ and $100$ or as $tip = percentage \times bill$ if percentage ranges between $0$ and $1$.

\item Calculate the total bill as $total\_bill = bill+tip$

\item Display to the user:
    \begin{itemize}
        \item The original bill (before the tip)
        \item The tip amount
        \item The total bill (including the tip)
    \end{itemize}

\item Ask the user how many people will be splitting the total bill (remember to save the user's input into a variable).

\item Calculate each person's share. For example if you saved the number of people splitting the bill into a variable
  called $numPeople$, then you would calculate each person's share as $share = \frac{total\_bill}{numPeople}$. You
  should think (or rethink) about the data types you are using and whether a data conversion is required at this point
  in your program.

\item Display to the user an exit message that is suitable for an academic setting.

\end{enumerate}

\noindent
A sample run of this program is shown below:
\begin{verbatim}
      jjumadinova@aldenv5:~/lab3$ javac Lab3.java
      jjumadinova@aldenv5:~/lab3$ java Lab3
      Janyl Jumadinova
      Lab 3
      Wed Sept 17 13:15:39 EDT 2014

      Please enter your name: Janyl
      Janyl, welcome to the Tip Calculator!
      Please enter the amount of your bill: 50
      Please enter the percentage that you want to tip: 15
      Your original bill was $50
      Your tip amount is $7.5
      Your total bill is $57.50
      How many people will be splitting the bill? 2
      Each person should pay $28.75
      Have a nice day! Thank you for using our service.

\end{verbatim}
\vspace{-0.2in}
Some points to remember as your complete this assignment:
\begin{itemize}
\item You will need to use the {\tt Scanner} class. Don't forget to import it at the top of your program. You may refer to
  class examples for code samples demonstrating the use of the {\tt Scanner}.
\item Think carefully about what data type you want to use for your variables.
\item Your program only needs to have one {\tt main} method.
\item To display a dollar sign you can just type ``\$'' inside the string of your output statement.
\item Note that your program will alternate between printing and computing---this
is okay!
\end{itemize}

\vspace{-0.2in}
\subsection*{Required Deliverables}

In addition to submitting signed and printed versions, for this assignment you are invited to submit electronic versions
of the following deliverables through the Bitbucket repository. As you complete this step, you should make sure that you
created a {\tt lab3/} directory within the Git repository.  Then, you can save all of the required deliverables in the
{\tt lab3/} directory---please see the course instructor or a teaching assistant if you are not able to create your
directory properly.

\begin{enumerate}

        \item A completed, properly commented, and formatted {\tt Lab3.java} program. Please make sure that your program
          prints your name, the lab number, and the date as the first few output lines of every program you write, and
          that it includes the comment header file with the Honor Code, your name, date, and the description of the
          program.

        \item \textbf{Three} outputs from running {\tt Lab3} in the terminal window three times with three different
          user inputs for the bill, tip, and the number of people splitting the bill. You may use {\tt gvim} to save all
          three of your outputs as follows: using the mouse, select everything from the ``{\tt java Lab3}'' command to
          the end of your output.  Right-click on the selected text and copy it.  Type ``{\tt gvim output}''---note that
          this {\em not} a Java program!---and use the ``Edit/Paste'' menu item to paste your program's output into the
          file.  Now, use ``{\tt :w}'' or the ``File/Save'' menu item to save this file. Please see the course
          instructor if you cannot save your output files.

\end{enumerate}
\vspace{-0.1in}
Share your program and the output file with me through your Git repository by correctly using ``{\tt git add}'', ``{\tt
git commit}'', and ``{\tt git push}'' commands. When you are done, please ensure that the Bitbucket Web site has
a {\tt lab3/} directory in your repository with the two files called {\tt Lab3.java} and {\tt output}. You should see
the instructor if you have questions about assignment submission.

\vspace{-0.1in}
\subsection*{A Special Challenge}

You may decide to try to format your floating-point-valued output to contain only a certain number of decimal places.
Read ahead to Section 3.6 in Chapter 3 to see how you can do this---or ask your instructor or a teaching assistant for
some suggestions for completing this challenge!

\end{document}

% CS 111 style
% Typical usage (all UPPERCASE items are optional):
%       \input 111pre
%       \begin{document}
%       \MYTITLE{Title of document, e.g., Lab 1\\Due ...}
%       \MYHEADERS{short title}{other running head, e.g., due date}
%       \PURPOSE{Description of purpose}
%       \SUMMARY{Very short overview of assignment}
%       \DETAILS{Detailed description}
%         \SUBHEAD{if needed} ...
%         \SUBHEAD{if needed} ...
%          ...
%       \HANDIN{What to hand in and how}
%       \begin{checklist}
%       \item ...
%       \end{checklist}
% There is no need to include a "\documentstyle."
% However, there should be an "\end{document}."
%
%===========================================================
\documentclass[11pt,twoside,titlepage]{article}
%%NEED TO ADD epsf!!
\usepackage{threeparttop}
\usepackage{graphicx}
\usepackage{latexsym}
\usepackage{color}
\usepackage{listings}
\usepackage{fancyvrb}
%\usepackage{pgf,pgfarrows,pgfnodes,pgfautomata,pgfheaps,pgfshade}
\usepackage{tikz}
\usepackage[normalem]{ulem}
\tikzset{
    %Define standard arrow tip
%    >=stealth',
    %Define style for boxes
    oval/.style={
           rectangle,
           rounded corners,
           draw=black, very thick,
           text width=6.5em,
           minimum height=2em,
           text centered},
    % Define arrow style
    arr/.style={
           ->,
           thick,
           shorten <=2pt,
           shorten >=2pt,}
}
\usepackage[noend]{algorithmic}
\usepackage[noend]{algorithm}
\newcommand{\bfor}{{\bf for\ }}
\newcommand{\bthen}{{\bf then\ }}
\newcommand{\bwhile}{{\bf while\ }}
\newcommand{\btrue}{{\bf true\ }}
\newcommand{\bfalse}{{\bf false\ }}
\newcommand{\bto}{{\bf to\ }}
\newcommand{\bdo}{{\bf do\ }}
\newcommand{\bif}{{\bf if\ }}
\newcommand{\belse}{{\bf else\ }}
\newcommand{\band}{{\bf and\ }}
\newcommand{\breturn}{{\bf return\ }}
\newcommand{\mod}{{\rm mod}}
\renewcommand{\algorithmiccomment}[1]{$\rhd$ #1}
\newenvironment{checklist}{\par\noindent\hspace{-.25in}{\bf Checklist:}\renewcommand{\labelitemi}{$\Box$}%
\begin{itemize}}{\end{itemize}}
\pagestyle{threepartheadings}
\usepackage{url}
\usepackage{wrapfig}
% \usepackage{hyperref}
\usepackage[hidelinks]{hyperref}
%=========================
% One-inch margins everywhere
%=========================
\setlength{\topmargin}{0in}
\setlength{\textheight}{8.5in}
\setlength{\oddsidemargin}{0in}
\setlength{\evensidemargin}{0in}
\setlength{\textwidth}{6.5in}
%===============================
%===============================
% Macro for document title:
%===============================
\newcommand{\MYTITLE}[1]%
   {\begin{center}
     \begin{center}
     \bf
     CMPSC 111\\Introduction to Computer Science I\\
     Fall 2014\\
     \medskip
     \end{center}
     \bf
     #1
     \end{center}
}
%================================
% Macro for headings:
%================================
\newcommand{\MYHEADERS}[2]%
   {\lhead{#1}
    \rhead{#2}
    \immediate\write16{}
    \immediate\write16{DATE OF HANDOUT?}
    \read16 to \dateofhandout
    \lfoot{\sc Handed out on \dateofhandout}
    \immediate\write16{}
    \immediate\write16{HANDOUT NUMBER?}
    \read16 to\handoutnum
    \rfoot{Handout \handoutnum}
   }

%================================
% Macro for bold italic:
%================================
\newcommand{\bit}[1]{{\textit{\textbf{#1}}}}

%=========================
% Non-zero paragraph skips.
%=========================
\setlength{\parskip}{1ex}

%=========================
% Create various environments:
%=========================
\newcommand{\PURPOSE}{\par\noindent\hspace{-.25in}{\bf Purpose:\ }}
\newcommand{\SUMMARY}{\par\noindent\hspace{-.25in}{\bf Summary:\ }}
\newcommand{\DETAILS}{\par\noindent\hspace{-.25in}{\bf Details:\ }}
\newcommand{\HANDIN}{\par\noindent\hspace{-.25in}{\bf Hand in:\ }}
\newcommand{\SUBHEAD}[1]{\bigskip\par\noindent\hspace{-.1in}{\sc #1}\\}
%\newenvironment{CHECKLIST}{\begin{itemize}}{\end{itemize}}

\begin{document}

\MYTITLE{Lab 5\\23 September 2015\\
Due Wednesday, 30 September by 2:30pm }

\subsection*{Objectives}
\vspace{-0.05in}

In addition to practicing the skills that you have learned in the past laboratory assignments, the purpose of this
assignment is to further explore the ideas of a ``class'' and an ``object'' in the Java programming language.
Additionally, you will learn how to use the methods provided by the {\tt java.lang.String} class to inspect and
manipulate a {\tt String} object. You will apply your knowledge of {\tt String}s to the application domain of
steganography, the practice of ``hiding'' messages inside of non-secret content. Finally, you will work in a team to
complete a real-world programming task, using tools like Git and Slack to support your collaborative completion of the
assignment.

\subsection*{General Guidelines for Labs}
\vspace{-0.05in}
\begin{itemize}
\item
{\bf Work on the Alden Hall computers.} If you want to work on a different
machine, be sure to transfer your programs to the Alden
machines and re-run them before submitting.
\item
{\bf Update your repository often!} You should add, commit,
and push your updated files each time you work on them.  I will not grade
your programs until the due date has passed.
\item
{\bf Review the Honor Code policy.} You
may discuss programs with others, but programs that are nearly identical
to others will be taken as evidence of violating the Honor Code.
\end{itemize}

\vspace{-0.1in}
\subsection*{Reading Assignment}
\vspace{-0.05in}

To review that you have already learned about about variables, expressions, and user input, please review Sections
2.1--2.6 in your textbook; pay close attention to the {\tt Scanner} methods in Figure 2.7 and the program in Listing
2.8. To learn more about Java classes and objects and, in particular, the methods provided by the {\tt java.lang.String}
class, please review Sections 3.1--3.2 in the textbook. Finally, don't forget to examine the ``Git Cheatsheet'' if you
have questions about how to create and use a new Git repository. Please see the instructor if you have questions
about these readings.

\vspace{-0.05in}
\subsection*{Creating Your Team's Collaboration Methods}
\vspace{-0.05in}

For this assignment, please pick a partner---in particular, an individual with whom you have not already worked on an
in-class assignment. You and your partner are responsible for implementing a complete solution to a real-world problem
involving the storage and transmission of secret messages using steganography. To start, you and your partner should
create a new Git repository hosted by Bitbucket. This new repository must adhere to the naming convention {\tt
<first-partner-username>-<second-partner-username>-cs111F2015-lab5}. After creating the repository with the correct
name, please share it with each other and with the course instructor. Now, make sure that you can both access this
repository through Git commands such as {\tt clone}, {\tt pull}, {\tt add}, {\tt commit}, and {\tt push}. Additionally,
you should make your own {\tt lab5} directory if your personal repository and make sure that, at the end of the
assignment, all of the files that you collaboratively created are also available in this directory. Please see the
course instructor or one of the teaching assistants if you cannot complete these first two steps.

\subsection*{Collaboratively Implementing a Steganography Program}
\vspace{-0.05in}

Within the field of computer security, there are many sub-fields that develop strategies for sending secret messages.
Involving the hiding of a secret ``in plain sight'', steganography is one way in which you can store and send a secret
message.

\vspace*{-.1in}
\subsection*{Required Deliverables}

In addition to submitting signed and printed versions of all materials, for this assignment you are invited to submit
electronic versions of the following deliverables through the Bitbucket repository. As you complete this step, you
should make sure that you created a {\tt lab5/} directory within your own Git repository.  Then, you can save all of the
required deliverables in the {\tt lab5/} directory---please see the course instructor or a teaching assistant if you are
not able to create your directory properly. Additionally, you and your partner should have created a separate repository
to support the completion of this assignment; it should also contain a {\tt lab5/} directory with all of the
deliverables that you collaboratively created for this assignment. Finally, please make sure that your team creates a
Slack channel that you can use to support all of your communication for this assignment.

\begin{enumerate}

        \item A completed, properly commented and formatted {\tt WordHide.java} program. Please make sure that your Java
          programs include the comment header file with the Honor Code, your name, date, and the description of the
          program.

        \item Three outputs from running {\tt WordHide} in the terminal window three times with three different user
          inputs from the user. Your output should clearly demonstrate how your program hides the user's input somewhere
          inside of the $20 \times 20$ grid of characters.  You may use {\tt gvim} to save all three of your outputs as
          follows: using the mouse, select everything from the ``{\tt java Lab3}'' command to the end of your output.
          Right-click on the selected text and copy it.  Type ``{\tt gvim output}''---note that this {\em not} a Java
          program!---and use the ``Edit/Paste'' menu item to paste your program's output into the file.  Now, use ``{\tt
          :w}'' or the ``File/Save'' menu item to save this file. Please see the course instructor if you cannot save
          your output files.

        \item A document called {\tt team\_roles} that describes, in detail, the tasks that each member of your
          team completed. Whenever possible, you should state the due date for the task and whether or not the team
          member actually succeeded in completing the task by the stated deadline.

\end{enumerate}

\vspace{-0.1in}

Share your program and the output file with me through your Git repository by correctly using ``{\tt git add}'', ``{\tt
git commit}'', and ``{\tt git push}'' commands. When you are done, please ensure that the Bitbucket Web site has a {\tt
lab4/} directory in your repository with only the three files called {\tt WordHide.java}, {\tt output}, and {\tt
team\_roles}. Remember, you should ultimately have the three files for this laboratory assignment in both your own
personal Git repository and the special repository that you created specifically for the completion of this assignment.
You should review the ``Git Cheatsheet'' and see the instructor if you have questions about assignment submission with
{\tt git}.

\end{document}

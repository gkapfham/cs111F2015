% CS 111 style
% Typical usage (all UPPERCASE items are optional):
%       \input 111pre
%       \begin{document}
%       \MYTITLE{Title of document, e.g., Lab 1\\Due ...}
%       \MYHEADERS{short title}{other running head, e.g., due date}
%       \PURPOSE{Description of purpose}
%       \SUMMARY{Very short overview of assignment}
%       \DETAILS{Detailed description}
%         \SUBHEAD{if needed} ...
%         \SUBHEAD{if needed} ...
%          ...
%       \HANDIN{What to hand in and how}
%       \begin{checklist}
%       \item ...
%       \end{checklist}
% There is no need to include a "\documentstyle."
% However, there should be an "\end{document}."
%
%===========================================================
\documentclass[11pt,twoside,titlepage]{article}
%%NEED TO ADD epsf!!
\usepackage{threeparttop}
\usepackage{graphicx}
\usepackage{latexsym}
\usepackage{color}
\usepackage{listings}
\usepackage{fancyvrb}
%\usepackage{pgf,pgfarrows,pgfnodes,pgfautomata,pgfheaps,pgfshade}
\usepackage{tikz}
\usepackage[normalem]{ulem}
\tikzset{
    %Define standard arrow tip
%    >=stealth',
    %Define style for boxes
    oval/.style={
           rectangle,
           rounded corners,
           draw=black, very thick,
           text width=6.5em,
           minimum height=2em,
           text centered},
    % Define arrow style
    arr/.style={
           ->,
           thick,
           shorten <=2pt,
           shorten >=2pt,}
}
\usepackage[noend]{algorithmic}
\usepackage[noend]{algorithm}
\newcommand{\bfor}{{\bf for\ }}
\newcommand{\bthen}{{\bf then\ }}
\newcommand{\bwhile}{{\bf while\ }}
\newcommand{\btrue}{{\bf true\ }}
\newcommand{\bfalse}{{\bf false\ }}
\newcommand{\bto}{{\bf to\ }}
\newcommand{\bdo}{{\bf do\ }}
\newcommand{\bif}{{\bf if\ }}
\newcommand{\belse}{{\bf else\ }}
\newcommand{\band}{{\bf and\ }}
\newcommand{\breturn}{{\bf return\ }}
\newcommand{\mod}{{\rm mod}}
\renewcommand{\algorithmiccomment}[1]{$\rhd$ #1}
\newenvironment{checklist}{\par\noindent\hspace{-.25in}{\bf Checklist:}\renewcommand{\labelitemi}{$\Box$}%
\begin{itemize}}{\end{itemize}}
\pagestyle{threepartheadings}
\usepackage{url}
\usepackage{wrapfig}
% \usepackage{hyperref}
\usepackage[hidelinks]{hyperref}
%=========================
% One-inch margins everywhere
%=========================
\setlength{\topmargin}{0in}
\setlength{\textheight}{8.5in}
\setlength{\oddsidemargin}{0in}
\setlength{\evensidemargin}{0in}
\setlength{\textwidth}{6.5in}
%===============================
%===============================
% Macro for document title:
%===============================
\newcommand{\MYTITLE}[1]%
   {\begin{center}
     \begin{center}
     \bf
     CMPSC 111\\Introduction to Computer Science I\\
     Fall 2014\\
     \medskip
     \end{center}
     \bf
     #1
     \end{center}
}
%================================
% Macro for headings:
%================================
\newcommand{\MYHEADERS}[2]%
   {\lhead{#1}
    \rhead{#2}
    \immediate\write16{}
    \immediate\write16{DATE OF HANDOUT?}
    \read16 to \dateofhandout
    \lfoot{\sc Handed out on \dateofhandout}
    \immediate\write16{}
    \immediate\write16{HANDOUT NUMBER?}
    \read16 to\handoutnum
    \rfoot{Handout \handoutnum}
   }

%================================
% Macro for bold italic:
%================================
\newcommand{\bit}[1]{{\textit{\textbf{#1}}}}

%=========================
% Non-zero paragraph skips.
%=========================
\setlength{\parskip}{1ex}

%=========================
% Create various environments:
%=========================
\newcommand{\PURPOSE}{\par\noindent\hspace{-.25in}{\bf Purpose:\ }}
\newcommand{\SUMMARY}{\par\noindent\hspace{-.25in}{\bf Summary:\ }}
\newcommand{\DETAILS}{\par\noindent\hspace{-.25in}{\bf Details:\ }}
\newcommand{\HANDIN}{\par\noindent\hspace{-.25in}{\bf Hand in:\ }}
\newcommand{\SUBHEAD}[1]{\bigskip\par\noindent\hspace{-.1in}{\sc #1}\\}
%\newenvironment{CHECKLIST}{\begin{itemize}}{\end{itemize}}

\begin{document}
\MYTITLE{Exam 1 Study Guide \\ Delivered: Wednesday, October 28, 2015 \\ Exam 1: Friday, November 6, 2015, 11:00 am}

\section*{Introduction}

This course will have its first exam on Friday, September 25, 2015 from 11:00 to 11:50 am. The exam will be ``closed
notes'' and ``closed book'' and it will cover the following materials. Please review the ``Course Schedule'' on the Web
site for the course to see the content and slides that we have covered to this date. Students may post questions about
this material to our Slack channel.

\begin{itemize}

  \itemsep 0in

  \item Chapter One in Lewis \& Loftus (e.g., ``Introduction to Computation and Programming'')

  \item Chapter Two in Lewis \& Loftus, Sections 2.1--2.9 (e.g., ``Data and Expressions'')

  \item Chapter Three in Lewis \& Loftus, Sections 3.1--3.7 (e.g., ``Using Classes and Objects'')

  \item Chapter Four in Lewis \& Loftus, Sections 4.1--4.9 (e.g., ``Writing Classes'')

  \item Chapter Five in Lewis \& Loftus, Sections 5.1--5.6 (e.g., ``Conditionals and Loops'')

  \item Using the many commands in the Linux operating system; editing in {\tt gvim}, compiling and executing
    programs in Linux; knowledge of the basic commands for using {\tt git} and Bitbucket.

  \item Your class notes and lecture slides, labs 1--8, practicals 1--8, and the handouts from lab.

\end{itemize}

\noindent The exam will be a mix of questions that have a form such as fill in the blank, short answer, true/false, and
completion.  The emphasis will be on the following topics:

\vspace*{-.05in}
\begin{itemize}

  \itemsep 0in

  \item Fundamental concepts in computing and the Java language (e.g., definitions and background).

  \item Fundamental concepts in programming languages (e.g., conditional logic and iteration).

  \item Practical laboratory techniques (e.g., editing, compiling, and running programs; effectively using files and
    directories; correctly using Bitbucket through the command-line {\tt git} program).

  \item Understanding Java programs (e.g., given a short, perhaps even one line, source code segment written in Java,
    understand what it does and be able to precisely describe its output).

  \item Composing Java statements and programs, given a description of what should be done. Students should be completely
    comfortable writing short source code statements that are in nearly-correct form as Java code. While your program may
    contain small syntactic errors, it is not acceptable to ``make up'' features of the Java programming language that do
    not exist in the language itself---so, please do not call a ``{\tt solveQuestionThree()}'' method!

\end{itemize}

\noindent No partial credit will be given for questions that are true/false, completion, or fill in the blank. Minimal
partial credit may be awarded for the questions that require a student to write a short answer. You are strongly
encouraged to write short, precise, and correct responses to all of the questions. When you are taking the exam, you
should do so as a ``point maximizer'' who first responds to the questions that you are most likely to answer correctly
for full points. Please make sure that you review the past quiz so that you can comfortably answer its questions.
Students should keep the time limitation in mind as you are absolutely required to submit the examination at the end of
the class period unless you have written permission for extra time from a member of the Learning Commons.  Students who
do not submit their exam on time will have their overall point total reduced.  Please see the course instructor if you
have questions about these policies.

\vspace*{-.15in}
\section*{Reminder Concerning the Honor Code}

\noindent Students are required to fully adhere to the Honor Code during the completion of this exam. More details about
the Allegheny College Honor Code are provided on the syllabus. Students are strongly encouraged to carefully review the
full statement of the Honor Code before taking this exam.

\noindent The following provides you with a review of Honor Code statement from the course syllabus:

The Academic Honor Program that governs the entire academic program at Allegheny College is described in the Allegheny
Academic Bulletin.  The Honor Program applies to all work that is submitted for academic credit or to meet non-credit
requirements for graduation at Allegheny College.  This includes all work assigned for this class (e.g., examinations,
laboratory assignments, and the final project).  All students who have enrolled in the College will work under the Honor
Program.  Each student who has matriculated at the College has acknowledged the following pledge:

\vspace*{-.11in}
\begin{quote}
  I hereby recognize and pledge to fulfill my responsibilities, as defined in the Honor Code, and to maintain the
  integrity of both myself and the College community as a whole.
\end{quote}
\vspace*{-.11in}

\noindent Students who have questions about Allegheny College's Honor Code and how it applies to the completion of a
quiz or an examination in Computer Science 111, should immediately schedule a meeting with the course instructor to
openly discuss their concerns.

% \noindent It is understood that an important part of the learning process in any course, and particularly one in
% computer science, derives from thoughtful discussions with teachers and fellow students.  Such dialogue is encouraged.
% However, it is necessary to distinguish carefully between the student who discusses the principles underlying a problem
% with others and the student who produces assignments that are identical to, or merely variations on, someone else's
% work.  While it is acceptable for students in this class to discuss their programs, data sets, and reports with their
% classmates, deliverables that are nearly identical to the work of others will be taken as evidence of violating the
% \mbox{Honor Code}.

\vspace*{-.15in}
\section*{Detailed Review of Content}
\vspace*{-.1in}

The listing of topics in the following subsections is not exhaustive; rather, it serves to illustrate the types of
concepts that students should study as they prepare for the exam. Please see the course instructor during office hours
if you have questions about any of the content listed in this section.

\vspace*{-.1in}
\subsection*{Chapter One}

\begin{itemize}

  \itemsep 0in
  \item Basic understanding of computer hardware and software
  \item Computer number systems (e.g., binary and decimal)
  \item Purpose for and steps of the fetch-decode-execute cycle in the CPU
  \item Layout of and access techniques for computer memory
  \item Knowledge of computer networking methods and programs
  \item Basic syntax and semantics of the Java programming language
  \item Input(s) and output(s) of the Java compiler and virtual machine

\end{itemize}

\vspace*{-.2in}
\subsection*{Chapter Two}
\vspace*{-.1in}

\begin{itemize}

  \itemsep -.015in
  \item Using escape sequences in the output of Java programs
  \item Ways to perform input and output in a Java program
  \item The variety of data types available to Java programmers
  \item The declaration of and assignment of values to variables
  \item Operators and operator precedence in Java expressions
  \item Techniques for converting variables from one data type to another
  \item Computer graphics and related topics such as pixels and screen resolution
  \item The use of the RGB system for specifying colors in Java programs

\end{itemize}

\vspace*{-.25in}
\subsection*{Chapter Three}
\vspace*{-.1in}

\begin{itemize}

  \itemsep -.015in
  \item The steps for creating a new instance of a Java class
  \item How to use technical diagrams to visualize and object in memory
  \item The meaning of the term ``alias'' in a Java program
  \item The creation and use of Strings in the Java language
  \item The ways in which Java packages promote high-quality programs
  \item The variety of ways in which you can create and use random numbers
  \item How to call and use the methods provided by the {\tt Math} class
  \item Ways in which programs create formatted output in a terminal window

\end{itemize}

\vspace*{-.25in}
\subsection*{Chapter Four}
\vspace*{-.1in}

\begin{itemize}

  \itemsep -.015in
  \item The steps for creating a new instance of a Java class
  \item How to use technical diagrams to visualize and object in memory
  \item The meaning of the term ``alias'' in a Java program
  \item The creation and use of Strings in the Java language
  \item The ways in which Java packages promote high-quality programs
  \item The variety of ways in which you can create and use random numbers
  \item How to call and use the methods provided by the {\tt Math} class
  \item Ways in which programs create formatted output in a terminal window

\end{itemize}

\vspace*{-.25in}
\subsection*{Chapter Four}
\vspace*{-.1in}

\begin{itemize}

  % \itemsep 0in
  \itemsep -.015in
  \item The steps for creating a new instance of a Java class
  \item How to use technical diagrams to visualize and object in memory
  \item The meaning of the term ``alias'' in a Java program
  \item The creation and use of Strings in the Java language
  \item The ways in which Java packages promote high-quality programs
  \item The variety of ways in which you can create and use random numbers
  \item How to call and use the methods provided by the {\tt Math} class
  \item Ways in which programs create formatted output in a terminal window

\end{itemize}



\end{document}

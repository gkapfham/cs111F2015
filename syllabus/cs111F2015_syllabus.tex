%!TEX root=cs111F2015-syllabus.tex

% CS 580 style
% Typical usage (all UPPERCASE items are optional):
%       \input 580pre
%       \begin{document}
%       \MYTITLE{Title of document, e.g., Lab 1\\Due ...}
%       \MYHEADERS{short title}{other running head, e.g., due date}
%       \PURPOSE{Description of purpose}
%       \SUMMARY{Very short overview of assignment}
%       \DETAILS{Detailed description}
%         \SUBHEAD{if needed} ...
%         \SUBHEAD{if needed} ...
%          ...
%       \HANDIN{What to hand in and how}
%       \begin{checklist}
%       \item ...
%       \end{checklist}
% There is no need to include a "\documentstyle."
% However, there should be an "\end{document}."
%
%===========================================================
\documentclass[11pt,twoside,titlepage]{article}
%%NEED TO ADD epsf!!
\usepackage{threeparttop}
\usepackage{graphicx}
\usepackage{latexsym}
\usepackage{color}
\usepackage{listings}
\usepackage{fancyvrb}
%\usepackage{pgf,pgfarrows,pgfnodes,pgfautomata,pgfheaps,pgfshade}
\usepackage{tikz}
\usepackage[normalem]{ulem}
\tikzset{
    %Define standard arrow tip
%    >=stealth',
    %Define style for boxes
    oval/.style={
           rectangle,
           rounded corners,
           draw=black, very thick,
           text width=6.5em,
           minimum height=2em,
           text centered},
    % Define arrow style
    arr/.style={
           ->,
           thick,
           shorten <=2pt,
           shorten >=2pt,}
}
\usepackage[noend]{algorithmic}
\usepackage[noend]{algorithm}
\newcommand{\bfor}{{\bf for\ }}
\newcommand{\bthen}{{\bf then\ }}
\newcommand{\bwhile}{{\bf while\ }}
\newcommand{\btrue}{{\bf true\ }}
\newcommand{\bfalse}{{\bf false\ }}
\newcommand{\bto}{{\bf to\ }}
\newcommand{\bdo}{{\bf do\ }}
\newcommand{\bif}{{\bf if\ }}
\newcommand{\belse}{{\bf else\ }}
\newcommand{\band}{{\bf and\ }}
\newcommand{\breturn}{{\bf return\ }}
\newcommand{\mod}{{\rm mod}}
\renewcommand{\algorithmiccomment}[1]{$\rhd$ #1}
\newenvironment{checklist}{\par\noindent\hspace{-.25in}{\bf Checklist:}\renewcommand{\labelitemi}{$\Box$}%
\begin{itemize}}{\end{itemize}}
\pagestyle{threepartheadings}
\usepackage{url}
\usepackage{wrapfig}
% removing the standard hyperref to avoid the horrible boxes
%\usepackage{hyperref}
\usepackage[hidelinks]{hyperref}
% added in the dtklogos for the bibtex formatting
\usepackage{dtklogos}
%=========================
% One-inch margins everywhere
%=========================
\setlength{\topmargin}{0in}
\setlength{\textheight}{8.5in}
\setlength{\oddsidemargin}{0in}
\setlength{\evensidemargin}{0in}
\setlength{\textwidth}{6.5in}
%===============================
%===============================
% Macro for document title:
%===============================
\newcommand{\MYTITLE}[1]%
   {\begin{center}
     \begin{center}
     \bf
     CMPSC 111\\Introduction to Computer Science I\\
     Fall 2015\\
     \medskip
     \end{center}
     \bf
     #1
     \end{center}
}
%================================
% Macro for headings:
%================================
\newcommand{\MYHEADERS}[2]%
   {\lhead{#1}
    \rhead{#2}
    %\immediate\write16{}
    %\immediate\write16{DATE OF HANDOUT?}
    %\read16 to \dateofhandout
    \def \dateofhandout {August 27, 2014}
    \lfoot{\sc Handed out on \dateofhandout}
    %\immediate\write16{}
    %\immediate\write16{HANDOUT NUMBER?}
    %\read16 to\handoutnum
    \def \handoutnum {1}
    \rfoot{Handout \handoutnum}
   }

%================================
% Macro for bold italic:
%================================
\newcommand{\bit}[1]{{\textit{\textbf{#1}}}}

%=========================
% Non-zero paragraph skips.
%=========================
\setlength{\parskip}{1ex}

%=========================
% Create various environments:
%=========================
\newcommand{\PURPOSE}{\par\noindent\hspace{-.25in}{\bf Purpose:\ }}
\newcommand{\SUMMARY}{\par\noindent\hspace{-.25in}{\bf Summary:\ }}
\newcommand{\DETAILS}{\par\noindent\hspace{-.25in}{\bf Details:\ }}
\newcommand{\HANDIN}{\par\noindent\hspace{-.25in}{\bf Hand in:\ }}
\newcommand{\SUBHEAD}[1]{\bigskip\par\noindent\hspace{-.1in}{\sc #1}\\}
%\newenvironment{CHECKLIST}{\begin{itemize}}{\end{itemize}}


\usepackage[compact]{titlesec}

\begin{document}
\MYTITLE{Syllabus}
\MYHEADERS{Syllabus}{}

\vspace*{-.1in}
\subsection*{Course Instructor}
Dr.\ Gregory M.\ Kapfhammer\\
\noindent Office Location: Alden Hall 108 \\
\noindent Office Phone: +1 814-332-2880 \\
\noindent Email: \url{gkapfham@allegheny.edu} \\
\noindent Twitter: \url{@GregKapfhammer} \\
\noindent Web Site: \url{http://www.cs.allegheny.edu/sites/gkapfham/}

\subsection*{Instructor's Office Hours}

\begin{itemize}
  \itemsep 0em
  \item Monday: 1:00 pm -- 2:00 pm (10 minute time slots)
  \item Tuesday: 3:30 pm -- 5:00 pm (15 minute time slots)
  \item Wednesday: 10:00 am -- 11:00 noon (10 minute time slots) {\em and} \\ \hspace*{.8in}
    4:30 pm - 5:30 pm (10 minute time slots)
  \item Thursday: 10:00 am -- 11:00 noon (10 minute time slots) {\em and} \\ \hspace*{.8in}
    2:30 pm - 5:00 pm (15 minute time slots)
\end{itemize}

\vspace*{-.1in}

\noindent To schedule a meeting with me during my office hours, please visit my Web site and click the ``Schedule'' link
in the top right-hand corner. Now, you can browse my office hours or schedule an appointment by clicking the correct
link and then reserving an open time slot. Students are also encouraged to post appropriate questions to a channel in
Slack, which is available at \url{https://CMPSC111Fall2015.slack.com}, and monitored by the instructor and the teaching
assistants.

\subsection*{Course Meeting Schedule}

Lecture, Discussion, and Group Work Session: Monday and Wednesday 11:00 am -- 11:50 am \\
Practical Session: Friday 11:00 am -- 11:50 am \\
Laboratory Session: Wednesday, 2:30 pm -- 4:20 pm \\
Final Examination: Tuesday, December 15, 2015 at 9:00 am

\subsection*{Course Catalogue Description}

\begin{quote}

An introduction to the principles of computer science with an emphasis on algorithmic problem solving and the
realization of algorithms using a modern object-oriented programming language. Topics include algorithms, problem
solving, programming, classes, primitive data types and objects, control structures, arrays and vectors, principles of
object-oriented design and programming, and an introduction to graphics and graphical user interfaces. The course also
includes an overview of the discipline of computer science and a study of the social implications of computer use. May
serve as the laboratory course in the Natural Science Division's distribution requirement. One laboratory per week.
Prerequisite: Knowledge of elementary algebra.

\end{quote}

\subsection*{Course Objectives}

The process of implementing and evaluating correct and efficient software involves the application of many interesting
theories, techniques, and tools.  In addition to learning problem solving and computational thinking skills, this class
will teach students how to use, design, implement, and test algorithms in an object-oriented programming language.
Students will learn more about fundamental concepts such as data types, conditional logic, and iteration while also
discovering how to use single-dimension, multi-dimensional, and extendible arrays and to implement graphical
applications.  Students also will gain hands-on experience in the use, design, implementation, and testing of software
during the laboratory and practical sessions and a final project.  Along with learning more about how to effectively
work in a team of diverse software developers, students will enhance their ability to write and present ideas about software in
a clear, concise, and compelling fashion.  Students will also develop an understanding of the fascinating connections
between computer science and other disciplines in the social and natural sciences and the humanities.

\subsection*{Performance Objectives}

At the completion of this semester, students must have a strong grasp of the basics of the object-oriented programming
paradigm and an ever-deepening knowledge of topics like conditional logic, iteration, recursion, exceptions, and
graphics programming.  Also, students should be able to handle many of the important, yet accidental, aspects of
implementing programs in Java.  That is, students should be comfortable with the use of Vim as an integrated development
environment and understand both the purpose and use of shell environment variables such as the {\tt CLASSPATH}.
Students should have a toolkit of programming language constructs that they can use to respond to the challenges that
they encounter during the development and evaluation of software. Finally, students should demonstrate the ability to
use both in-person discussions and cutting-edge software tools to effectively communicate and collaborate with a group
of diverse team members.

\subsection*{Required Textbook}

\noindent{\em Java Software Solutions: Foundations of Program Design}. John Lewis and William Loftus,
Eighth Edition, ISBN:\ 978-0133594959, 806 pages, 2015. \\
(References to the textbook are abbreviated as ``JSS'' on the course Web site and the syllabus).

\noindent
Students who want to improve their technical writing skills may consult the following books.

\noindent{\em BUGS in Writing: A Guide to Debugging Your Prose}. Lyn Dupr\'e. Second Edition,  ISBN-10: 020137921X,
ISBN-13: 978-0201379211, 704 pages, 1998.

\noindent{\em Writing for Computer Science}.  Justin Zobel. Second Edition,  ISBN-10: 1852338024, ISBN-13:
978-1852338022, 270 pages, 2004.

\noindent
Along with reading the required textbook, you will be asked to study additional articles from a wide variety of
conference proceedings, journals, and the popular press.

\subsection*{Class Policies}

\subsubsection*{Grading}

The grade that a student receives in this class will be based on the following categories. All percentages are
approximate and, if the need to do so presents itself, it is possible for the course instructor to change the assigned
percentages during the academic semester.

\begin{center}
  \begin{tabular}{ll}
    Class Participation and Instructor Meetings & 10\% \\
    Midterm Examination                         & 15\% \\
    Final Examination                           & 15\% \\
    Quizzes                                     & 10\% \\
    Laboratory Assignments                      & 30\% \\
    Practical Assignments                       & 10\% \\
    Final Project                               & 10\%
  \end{tabular}
\end{center}

% \vspace*{-.1in}
\noindent
These grading categories have the following definitions:
\vspace*{-.1in}

\begin{itemize}

  \item {\em Class Participation and Instructor Meetings}: All students are required to actively participate during all
    of the class sessions. Your participation will take forms such as answering questions about the required reading
    assignments, asking constructive questions of group members, giving presentations, and leading a discussion session.
    Furthermore, all students are required to meet with the course instructor during office hours for at least fifteen
    minutes during the Fall 2015 semester.  These meetings must be scheduled through the course instructor's reservation
    system and documented on a meeting record that you submit on the day of the final examination. Finally, you must
    regularly participate in the discussions on the Slack channels for this course. A student will receive an interim
    and final grade for this category.

  \item {\em Quizzes}: The two quizzes will cover all of the material in their associated module(s).  While the second
    quiz is not cumulative, it will assume that a student has a basic understanding of the material that
    was the focus of the previous quiz.  The date for each of the quizzes will be announced at least one week in
    advance of the scheduled date.  Unless prior arrangements are made with the course instructor, all students will be
    expected to take these two quizzes on the scheduled date and complete the quizzes in the stated period of time.

  \item {\em Midterm Examination}: The midterm is an hour-long cumulative test covering all of the material from the
    class, practical, and laboratory sessions, as outlined on the review sheet. Unless prior arrangements are made with
    the course instructor, all students will be expected to take this test on the scheduled date and complete the test
    in the stated period of time.

  \item {\em Final Examination}: The final examination is a three-hour cumulative test.  By enrolling in this
    course, students agree that, unless there are severe extenuating circumstances, they will take the final examination
    at the date and time stated on the first page of the syllabus.

  \item {\em Laboratory Assignments}: These assignments invite students to explore different techniques for designing,
    implementing, evaluating, and documenting software solutions to challenging problems that often have a connection to
    real-world concerns.  Many of the assignments will require students to conduct experiments and collect, analyze, and
    write about data sets.  To best ensure that students are ready to develop software in both other classes at
    Allegheny College and after graduation, students will complete assignments both on an individual basis and in teams.
    When teamwork is required, the instructor will assign individuals to teams.

  \item {\em Final Project}: This project will furnish you with the description of a problem and ask you to design,
    implement, describe, and orally present a correct and carefully evaluated solution. Completion of the final project
    will require you to apply all of the knowledge and skills that you have acquired during the course of the semester
    to solve a problem and, whenever possible, make your solution and results publicly available in a free and open
    fashion.

\end{itemize}

\subsubsection*{Assignment Submission}

All assignments will have a stated due date. Electronic versions of the practical, laboratory, and final project
assignments must be submitted to the version control repository that the student creates at the start of the semester.
Additionally, the printed version of the assignment is to be turned in at the beginning of the class on that due date;
the printed materials must be dated and signed with the Honor Code pledge of the student(s) completing the work.  Late
assignments will be accepted for up to one week past the assigned due date with a 15\% penalty. All of the late
assignments must be turned in at the beginning of the session that is scheduled one week after the due date. Unless
special arrangements are made with the instructor, no work will be accepted after the late deadline. For any
assignment completed in a group, students must also turn in a one-page document that describes each group member's
contribution to the submitted deliverables.

\subsubsection*{Course Attendance}

It is mandatory for all students to attend all of the class, practical, and laboratory sessions. If, due to extenuating
circumstances, you will not be able to attend a session, then, whenever possible, please see the instructor at least one
week in advance to describe your situation.  Students who miss more than five unexcused sessions will have their final
grade in the course reduced by one letter grade. Students who miss more than ten of the aforementioned events will fail
the course.

% \subsection*{Laboratory Attendance Policy}
%
% In order to acquired the proper skills in technical writing, critical reading, and the presentation of technical
% material, it is essential for students to have hands-on experience in a laboratory. Therefore, it is mandatory for all
% students to attend the laboratory sessions. If you will not be able to attend a laboratory, then please see the course
% instructors at least one week in advance in order to explain your situation. Students who miss more than two unexcused
% laboratories will have their final grade in the course reduced by one letter grade.  Students who miss more than four
% unexcused laboratories will automatically fail the course.
%

\subsubsection*{Use of Laboratory Facilities}

Throughout the semester, we will investigate many different software tools that computer scientists use during the
design, implementation, and evaluation of algorithms.  The course instructor and the department's systems administrator
have invested a considerable amount of time to ensure that our laboratories support the completion of all of the
assignments and projects.  To this end, students are required to complete all of the laboratory and practical
assignments and the final project while using the department's laboratory facilities. The course instructor and the
systems administrator normally do not assist students in configuring their personal computers.

\subsubsection*{Class Preparation}

% The study of the computer science discipline is very challenging.  Students in this class will be challenged to learn
% the principles and practice of software development.  During the coming semester even the most diligent student will
% experience times of frustration when they are attempting to understand a challenging concept or complete a difficult
% laboratory assignment.  In many situations some of the material that we examine will initially be confusing : do not
% despair!  Press on and persevere!

In order to minimize confusion and maximize learning, students must invest time to prepare for the class discussions,
lectures, and practical sessions.  During the class periods, the course instructor will often pose demanding questions
that could require group discussion, the creation of a program or data set, a vote on a thought-provoking issue, or a
group presentation.  Only students who have prepared for class by reading the assigned material and reviewing the
current laboratory and practical assignments will be able to effectively participate in these discussions.

More importantly, only prepared students will be able to acquire the knowledge and skills that are needed to be
successful in this course, subsequent courses, and the field of computer science.  In order to help students remain
organized and effectively prepare for classes, the course instructor will maintain a class schedule with reading
assignments and presentation slides.   During the class sessions students will also be required to download, use, and
modify programs and data sets that are made available through means such as the course Web site.

\subsubsection*{Seeking Assistance}

Students who are struggling to understand the knowledge and skills developed in a class, laboratory, or practical
session are encourage to seek assistance from the course instructor, the teaching assistants, or the departmental
tutors. Throughout the semester, students should, within the bounds of the Honor Code, ask and answer questions on the
Slack site for our course; please request assistance from the instructor, teaching assistants, and tutors first through
Slack before sending an email. Students who need the course instructor's assistance must schedule a meeting through his
course web site and come to the meeting with all of the details needed to discuss their question.

\subsubsection*{Using Email}

Although we will primarily use Slack for class communication, I will sometimes use email to send announcements about
important matters such as changes in the schedule. It is your responsibility to check your email at least once a day and to
ensure that you can reliably send and receive emails. This class policy is based on the statement about the use of email that
appears in {\em The Compass}, the College's student handbook; please see the instructor if you do not have this
handbook.

\subsubsection*{Honor Code}

The Academic Honor Program that governs the entire academic program at Allegheny College is described in the Allegheny
Course Catalogue.  The Honor Program applies to all work that is submitted for academic credit or to meet non-credit
requirements for graduation at Allegheny College.  This includes all work assigned for this class (e.g., examinations,
  laboratory assignments, and the final project).  All students who have enrolled in the College will work under the Honor
Program.  Each student who has matriculated at the College has acknowledged the following pledge:

\vspace*{-.11in}
\begin{quote}
  I hereby recognize and pledge to fulfill my responsibilities, as defined in the Honor Code, and to maintain the
  integrity of both myself and the College community as a whole.
\end{quote}
\vspace*{-.11in}

\noindent It is understood that an important part of the learning process in any course, and particularly one in
computer science, derives from thoughtful discussions with teachers and fellow students.  Such dialogue is encouraged.
However, it is necessary to distinguish carefully between the student who discusses the principles underlying a problem
with others and the student who produces assignments that are identical to, or merely variations on, someone else's
work.  While it is acceptable for students in this class to discuss their programs, data sets, and reports with their
classmates, deliverables that are nearly identical to the work of others will be taken as evidence of violating the
\mbox{Honor Code}.


\subsubsection*{Disability Services}

The Americans with Disabilities Act (ADA) is a federal anti-discrimination statute that provides comprehensive civil
rights protection for persons with disabilities.  Among other things, this legislation requires all students with
disabilities be guaranteed a learning environment that provides for reasonable accommodation of their disabilities.
Students with disabilities who believe they may need accommodations in this class are encouraged to contact Disability
Services at 332-2898.  Disability Services is part of the Learning Commons and is located in Pelletier Library.
Please do this as soon as possible to ensure that approved accommodations are implemented in a timely fashion.


\subsection*{Welcome to an Adventure in Computer Science}

In reference to software, Frederick P.\ Brooks, Jr.\ wrote in chapter one of {\em The Mythical Man Month}, ``The magic
of myth and legend has come true in our time.'' Software is a pervasive aspect of our society that changes how we think
and act.  Efficient and correct software also has the potential to positively influence the lives of many people.
Moreover, the design, implementation, evaluation, and documentation of software are exciting and rewarding activities!
At the start of this class, I invite you to pursue, with great enthusiasm and vigor, this adventure in computer science.

\end{document}

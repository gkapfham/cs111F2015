% CS 111 style
% Typical usage (all UPPERCASE items are optional):
%       \input 111pre
%       \begin{document}
%       \MYTITLE{Title of document, e.g., Lab 1\\Due ...}
%       \MYHEADERS{short title}{other running head, e.g., due date}
%       \PURPOSE{Description of purpose}
%       \SUMMARY{Very short overview of assignment}
%       \DETAILS{Detailed description}
%         \SUBHEAD{if needed} ...
%         \SUBHEAD{if needed} ...
%          ...
%       \HANDIN{What to hand in and how}
%       \begin{checklist}
%       \item ...
%       \end{checklist}
% There is no need to include a "\documentstyle."
% However, there should be an "\end{document}."
%
%===========================================================
\documentclass[11pt,twoside,titlepage]{article}
%%NEED TO ADD epsf!!
\usepackage{threeparttop}
\usepackage{graphicx}
\usepackage{latexsym}
\usepackage{color}
\usepackage{listings}
\usepackage{fancyvrb}
%\usepackage{pgf,pgfarrows,pgfnodes,pgfautomata,pgfheaps,pgfshade}
\usepackage{tikz}
\usepackage[normalem]{ulem}
\tikzset{
    %Define standard arrow tip
%    >=stealth',
    %Define style for boxes
    oval/.style={
           rectangle,
           rounded corners,
           draw=black, very thick,
           text width=6.5em,
           minimum height=2em,
           text centered},
    % Define arrow style
    arr/.style={
           ->,
           thick,
           shorten <=2pt,
           shorten >=2pt,}
}
\usepackage[noend]{algorithmic}
\usepackage[noend]{algorithm}
\newcommand{\bfor}{{\bf for\ }}
\newcommand{\bthen}{{\bf then\ }}
\newcommand{\bwhile}{{\bf while\ }}
\newcommand{\btrue}{{\bf true\ }}
\newcommand{\bfalse}{{\bf false\ }}
\newcommand{\bto}{{\bf to\ }}
\newcommand{\bdo}{{\bf do\ }}
\newcommand{\bif}{{\bf if\ }}
\newcommand{\belse}{{\bf else\ }}
\newcommand{\band}{{\bf and\ }}
\newcommand{\breturn}{{\bf return\ }}
\newcommand{\mod}{{\rm mod}}
\renewcommand{\algorithmiccomment}[1]{$\rhd$ #1}
\newenvironment{checklist}{\par\noindent\hspace{-.25in}{\bf Checklist:}\renewcommand{\labelitemi}{$\Box$}%
\begin{itemize}}{\end{itemize}}
\pagestyle{threepartheadings}
\usepackage{url}
\usepackage{wrapfig}
% \usepackage{hyperref}
\usepackage[hidelinks]{hyperref}
%=========================
% One-inch margins everywhere
%=========================
\setlength{\topmargin}{0in}
\setlength{\textheight}{8.5in}
\setlength{\oddsidemargin}{0in}
\setlength{\evensidemargin}{0in}
\setlength{\textwidth}{6.5in}
%===============================
%===============================
% Macro for document title:
%===============================
\newcommand{\MYTITLE}[1]%
   {\begin{center}
     \begin{center}
     \bf
     CMPSC 111\\Introduction to Computer Science I\\
     Fall 2014\\
     \medskip
     \end{center}
     \bf
     #1
     \end{center}
}
%================================
% Macro for headings:
%================================
\newcommand{\MYHEADERS}[2]%
   {\lhead{#1}
    \rhead{#2}
    \immediate\write16{}
    \immediate\write16{DATE OF HANDOUT?}
    \read16 to \dateofhandout
    \lfoot{\sc Handed out on \dateofhandout}
    \immediate\write16{}
    \immediate\write16{HANDOUT NUMBER?}
    \read16 to\handoutnum
    \rfoot{Handout \handoutnum}
   }

%================================
% Macro for bold italic:
%================================
\newcommand{\bit}[1]{{\textit{\textbf{#1}}}}

%=========================
% Non-zero paragraph skips.
%=========================
\setlength{\parskip}{1ex}

%=========================
% Create various environments:
%=========================
\newcommand{\PURPOSE}{\par\noindent\hspace{-.25in}{\bf Purpose:\ }}
\newcommand{\SUMMARY}{\par\noindent\hspace{-.25in}{\bf Summary:\ }}
\newcommand{\DETAILS}{\par\noindent\hspace{-.25in}{\bf Details:\ }}
\newcommand{\HANDIN}{\par\noindent\hspace{-.25in}{\bf Hand in:\ }}
\newcommand{\SUBHEAD}[1]{\bigskip\par\noindent\hspace{-.1in}{\sc #1}\\}
%\newenvironment{CHECKLIST}{\begin{itemize}}{\end{itemize}}

\begin{document}
\MYTITLE{Practical 2\\11 September 2015\\Due in Bitbucket by midnight of the day of your practical \\ ``Checkmark'' grade}

\subsection*{Summary}

Create a Java program that prints something ``interesting'', and then use the {\tt git add}, {\tt git commit}, and {\tt git
  push} commands to upload it to your Git repository hosted by Bitbucket.  See the end of the assignment for a few hints
and suggestions for creating Java programs that perform output.

\vspace*{-.1in}
\subsection*{Review the Textbook}

In addition to reviewing the slides for Chapter 2, be sure to read Section 2.1 of your book as it explains how
to print some of the special characters, a topic that is also discussed at the end of this assignment. Please see the
course instructor if you have questions about using escape characters.

\vspace*{-.1in}
\subsection*{Exercise: Print Something ``Interesting''}

Create a Java program that uses a sequence of {\em no more than ten} {\tt System.out.println} statements to print
something ``interesting.'' You may {\em not} use any other features of Java, such as variables, loops, etc.  However,
you are {\em required} to use at least one of the ``escaped'' characters, such as \verb$\"$ or \verb$\\$.  Remember, it
is possible to get pictures that are taller than ten lines by using the \verb$\n$ character in your {\tt println}
statements. Please see the instructor if you have questions about this requirement.

Your program must print your name and today's date (using ``{\tt new Date()}'' in a {\tt println} statement).  This will
not count as part of the ten required print statements.

% \begin{quote}
You must come up with an {\em original} design---{\em under no circumstances} should you copy a design from another
source, such as an ``ASCII Art'' web site. (However, you may look at such sites for inspiration.) The next two pages
furnish two examples that you are encouraged to try. As you complete these two examples, please take time to pause and
reflect on why the program produces the output that it does.  If you cannot get one of these examples to work, then
please show a teaching assistant or the course instructor the problem that you are encountering.
% \end{quote}

\subsection*{Preparing for the Programming Task}

Before you start creating the Java program required by this assignment, you should separately type the commands ``{\tt cd
  cs111F2015}'', ``{\tt cd cs111F2015-<your user name>}'', and ``{\tt cd practicals}'' in your terminal window. Once you
are in the {\tt practicals/} directory of your Git repository, you can type the command ``{\tt mkdir practical02}'' to
create a new directory for this assignment. You can run the {\tt gvim} command from this directory when you are ready to
begin to implement the required Java program. Please see the instructor if you have problems with these preparatory steps.

% \newpage

\subsection*{Example 1: File ``{\tt PrintName.java}''}
\begin{verbatim}
     //**********************************
     // Bob Roos
     // Practical 2, 11-12 September 2015
     //
     // Prints a name `Bob'
     //**********************************
     import java.util.Date;
     public class PrintName
     {
       public static void main(String[] args)
       {
          System.out.println("Bob Roos, CMPSC 111\n" + new Date() + "\n");
          System.out.println(" ____        __");
          System.out.println("  |  \\        |");
          System.out.println("  |__/   __   |__");
          System.out.println("  |  \\  /  \\  |  \\");
          System.out.println(" _|__/  \\__/ _|__/");
       }
     }
\end{verbatim}

\noindent{\bf OUTPUT:}
\begin{verbatim}
     javac PrintName.java
     java PrintName
     Bob Roos, CMPSC 111
     Wed Sep 10 21:04:41 EST 2015

      ____        __
       |  \        |
       |__/   __   |__
       |  \  /  \  |  \
      _|__/  \__/ _|__/
\end{verbatim}

\newpage

\subsection*{Example 2: File ``{\tt PrintFace.java}''}
\begin{verbatim}
     //**********************************
     // Janyl Jumadinova
     // Practical 2, 11-12 September 2015
     // Prints a face
     //**********************************
     import java.util.Date;
     public class PrintFace
     {
       public static void main(String[] args)
       {
          System.out.println("Janyl Jumadinova, CMPSC 111\n" + new Date() + "\n");
          System.out.println("   \\\\\\|||///");
          System.out.println("   /       \\");
          System.out.println("   | -- -- |");
          System.out.println("  @|  O O  |@");
          System.out.println("   |   V   |");
          System.out.println("    \\ \\_/ /");
          System.out.println("     \\___/");
       }
     }
\end{verbatim}
\noindent{\bf OUTPUT:}
\begin{verbatim}
     javac PrintFace.java
     java PrintFace
     Janyl Jumadinova, CMPSC 111
     Wed Sep 10 21:11:55 EST 2015

        \\\|||///
        /       \
        | -- -- |
       @|  O O  |@
        |   V   |
         \ \_/ /
          \___/
\end{verbatim}

% \newpage
\subsection*{Using Version Control Correctly}

As you are typing your program in the {\tt gvim} text editor, you should regularly save your files.  Once you have
created a preliminary version of your program and it compiles and runs as anticipated, you should use the ``{\tt git
add}'' command to stage it in your Git repository.  Next, you can use the ``{\tt git commit}'' command to save it in
your local repository with a version control message.  Finally, you can run ``{\tt git push}'' to transfer your file to
the Bitbucket servers.  For this practical assignment, you do {\em not} have to hand in a hard copy of anything---just
upload your Java program and an output produced by your program to Bitbucket using the appropriate {\tt git} commands.
Please review your ``Git Cheatsheet'' and talk with a member of the class, the course instructor, or a teaching
assistant if you do not understand how to use the Git version control system.

\vspace*{-.1in}
\subsection*{Hints About Java Programming and Escaped Characters}
\vspace*{-.05in}

The name of your program file (for instance, ``{\tt PrintFace.java}'') must be the same as the name in the ``{\tt public
  class ...}'' statement---see earlier examples where you practiced this skill.

The characters ``\verb$\$'' (backslash) and ``\verb$"$'' (double-quote)
require special handling. To print them out, you need to put an extra
``\verb$\$'' in front of them. For instance,

\vspace*{-.1in}
\begin{quote}
The statement:\ \ \ \ \ \ \ \verb$System.out.println("backslash: \\, quote: \"");$\\
prints:\ \ \ \ \ \ \ \ \ \ \ \ \ \ \ \ \ \
\verb$backslash: \, quote: "$
\end{quote}
\vspace*{-.1in}

Since this is one of our first practical assignments and you are still learning how to use the Java programming
language, don't become frustrated if you make a mistake. Instead, use your mistakes as an opportunity for learning both
about the necessary technology and the background and expertise of the other students in the class, the teaching
assistants, and the instructor.

\vspace*{-.2in}
\subsection*{General Guidelines for Practical Sessions}

% Please follow these general guidelines as you complete this practical assignment:

\vspace*{-.05in}
\begin{itemize}

% \itemsep -0.01in
\itemsep 0in

\item {\bf Experiment!} Practical sessions are for learning by doing without the pressure of grades or ``right/wrong''
  answers. So try things!  The best way to learn is by intelligently experimenting.

\item {\bf Submit \textbf{\textit{Something}}.} Your grade for this assignment is a ``checkmark'' indicating whether you
  did or did not complete the work and submit something to the Bitbucket repository using the ``{\tt git add}'', ``{\tt
    git commit}'', and ``{\tt git push}'' commands.

\item {\bf Practice Key Laboratory Skills.} As you are completing this assignment, practice using the {\tt gvim} text
  editor and the Ubuntu terminal until you can easily use their most important features.  Additionally, ask
  a teaching assistant or the course instructor to teach you some of the advanced features of {\tt gvim} and the
  terminal, thereby helping you to work more effectively.

\item {\bf Try to Finish During the Class Session.} Practical exercises are not intended to be the equal of the
  laboratory assignments. If you are simply a slow typist, I've given you until the end of the day, but ideally you
  should upload a file, even a partially working one, by the end of the class period. So, make sure that you correctly
  upload your file to your Git repository!

  % You also should ensure that, for this assignment and all subsequent assignments, you can confidently upload files to
  % your Git repository during the practical session.

\item {\bf Help One Another!} If your neighbor is struggling and you know what to do, offer your help. Don't ``do the
  work'' for them, but advise them on what to type or how to handle things. If you are stuck on a part of this practical
  session and you could not find any insights in either your textbook or online sources, formulate an intelligent
  question to ask your neighbor, a teaching assistant, or a course instructor. Try to strike the right balance between
  asking for help when you cannot solve a problem and working independently to find a solution.

\item {\bf Update Your Repository Often!} You should {\tt add}, {\tt commit}, and {\tt push} your updated files each
  time you work on them, always including descriptive messages about each code change.

\item {\bf Review the Honor Code Policy on the Syllabus.} Remember that while you may discuss programs with other
  students in the course, programs that are nearly identical to, or merely variations on, the work of others will be
  taken as evidence of violating the Honor Code.

\end{itemize}

\end{document}

% CS 111 style
% Typical usage (all UPPERCASE items are optional):
%       \input 111pre
%       \begin{document}
%       \MYTITLE{Title of document, e.g., Lab 1\\Due ...}
%       \MYHEADERS{short title}{other running head, e.g., due date}
%       \PURPOSE{Description of purpose}
%       \SUMMARY{Very short overview of assignment}
%       \DETAILS{Detailed description}
%         \SUBHEAD{if needed} ...
%         \SUBHEAD{if needed} ...
%          ...
%       \HANDIN{What to hand in and how}
%       \begin{checklist}
%       \item ...
%       \end{checklist}
% There is no need to include a "\documentstyle."
% However, there should be an "\end{document}."
%
%===========================================================
\documentclass[11pt,twoside,titlepage]{article}
%%NEED TO ADD epsf!!
\usepackage{threeparttop}
\usepackage{graphicx}
\usepackage{latexsym}
\usepackage{color}
\usepackage{listings}
\usepackage{fancyvrb}
%\usepackage{pgf,pgfarrows,pgfnodes,pgfautomata,pgfheaps,pgfshade}
\usepackage{tikz}
\usepackage[normalem]{ulem}
\tikzset{
    %Define standard arrow tip
%    >=stealth',
    %Define style for boxes
    oval/.style={
           rectangle,
           rounded corners,
           draw=black, very thick,
           text width=6.5em,
           minimum height=2em,
           text centered},
    % Define arrow style
    arr/.style={
           ->,
           thick,
           shorten <=2pt,
           shorten >=2pt,}
}
\usepackage[noend]{algorithmic}
\usepackage[noend]{algorithm}
\newcommand{\bfor}{{\bf for\ }}
\newcommand{\bthen}{{\bf then\ }}
\newcommand{\bwhile}{{\bf while\ }}
\newcommand{\btrue}{{\bf true\ }}
\newcommand{\bfalse}{{\bf false\ }}
\newcommand{\bto}{{\bf to\ }}
\newcommand{\bdo}{{\bf do\ }}
\newcommand{\bif}{{\bf if\ }}
\newcommand{\belse}{{\bf else\ }}
\newcommand{\band}{{\bf and\ }}
\newcommand{\breturn}{{\bf return\ }}
\newcommand{\mod}{{\rm mod}}
\renewcommand{\algorithmiccomment}[1]{$\rhd$ #1}
\newenvironment{checklist}{\par\noindent\hspace{-.25in}{\bf Checklist:}\renewcommand{\labelitemi}{$\Box$}%
\begin{itemize}}{\end{itemize}}
\pagestyle{threepartheadings}
\usepackage{url}
\usepackage{wrapfig}
% \usepackage{hyperref}
\usepackage[hidelinks]{hyperref}
%=========================
% One-inch margins everywhere
%=========================
\setlength{\topmargin}{0in}
\setlength{\textheight}{8.5in}
\setlength{\oddsidemargin}{0in}
\setlength{\evensidemargin}{0in}
\setlength{\textwidth}{6.5in}
%===============================
%===============================
% Macro for document title:
%===============================
\newcommand{\MYTITLE}[1]%
   {\begin{center}
     \begin{center}
     \bf
     CMPSC 111\\Introduction to Computer Science I\\
     Fall 2014\\
     \medskip
     \end{center}
     \bf
     #1
     \end{center}
}
%================================
% Macro for headings:
%================================
\newcommand{\MYHEADERS}[2]%
   {\lhead{#1}
    \rhead{#2}
    \immediate\write16{}
    \immediate\write16{DATE OF HANDOUT?}
    \read16 to \dateofhandout
    \lfoot{\sc Handed out on \dateofhandout}
    \immediate\write16{}
    \immediate\write16{HANDOUT NUMBER?}
    \read16 to\handoutnum
    \rfoot{Handout \handoutnum}
   }

%================================
% Macro for bold italic:
%================================
\newcommand{\bit}[1]{{\textit{\textbf{#1}}}}

%=========================
% Non-zero paragraph skips.
%=========================
\setlength{\parskip}{1ex}

%=========================
% Create various environments:
%=========================
\newcommand{\PURPOSE}{\par\noindent\hspace{-.25in}{\bf Purpose:\ }}
\newcommand{\SUMMARY}{\par\noindent\hspace{-.25in}{\bf Summary:\ }}
\newcommand{\DETAILS}{\par\noindent\hspace{-.25in}{\bf Details:\ }}
\newcommand{\HANDIN}{\par\noindent\hspace{-.25in}{\bf Hand in:\ }}
\newcommand{\SUBHEAD}[1]{\bigskip\par\noindent\hspace{-.1in}{\sc #1}\\}
%\newenvironment{CHECKLIST}{\begin{itemize}}{\end{itemize}}

\begin{document}

\MYTITLE{Practical 7 \\ 6--7 November 2015 \\ Due in Bitbucket by midnight of the day of your practical \\ ``Checkmark'' grade}

\subsection*{Summary}
\vspace*{-.05in}

In this practical you will write a Java program that will allow the user to guess a number. To complete this task you will use {\tt while} loops and {\tt if/else} statements.  Then, using the ``{\tt git add}'', ``{\tt git  commit}'', and ``{\tt git push}'' commands you will
upload your modified source code files and the output you obtain from running your program to your Git repository hosted
by Bitbucket.

\vspace*{-.1in}
\subsection*{Review the Textbook}
\vspace*{-.05in}

You may refer to sections 5.1--5.4 in your textbook to learn more about {\tt if/else} statements and {\tt while} loops.
You may also refer to the section in this document called ``The {\tt while} Loop Overview'' for an example of using a
{\tt while} loop in a Java program. To review the details about random number generation and the {\tt java.util.Random}
class, you can study Section 3.4. Please see the course instructor or a teaching assistant if you have questions about
the concepts underlying iteration or recursion or if you are unsure of how to realize these fundamental ideas in a Java
program.

\vspace*{-.1in}
\subsection*{Guessing Game}
\vspace*{-.05in}

Write a Java program that will play a guessing game with the user of the program.  The user must try to guess a number
between $1$ and $100$. If the user's guess is wrong, print out a helpful hint, such as ``that's too high'' or ``that's
too low''. Keep repeating this until the user guesses correctly.

Your Java program should complete the following tasks:

\begin{itemize}

  \item Print your name, practical number, and the date (so, remember to use ``{\tt new Date()}'').

  \item
    Declare and initialize a variable of type {\tt int} to be equal to some
    randomly generated number in the range between 1 and 100; this is the correct answer that the user will try to guess.

  \item
    \textbf{Repeatedly} ask the user to enter a number until the user is able to guess the correct answer.

  \item
    Print out an appropriate statement if the user's input is equal to the correct answer.

  \item
    Print out ``Too low!" or another appropriate statement if the number entered by the user is less than the correct answer.

  \item
    Print out ``Too high!" or another appropriate statement if the number entered by the user is greater than the correct answer.

  \item Print out how many tries it took the user to guess the correct answer.

\end{itemize}

\vspace*{-.1in}
\subsection*{Completing the Practical Assignment}

\vspace*{-.1in}
Create one Java file, called {\tt Practical7.java} for this assignment.
This class will contain the main method where you will implement a
solution to the problem outlined on the previous page.

\noindent To finish this assignment and earn a ``checkmark'', you should submit the
{\tt Practical7.java} file in your Bitbucket repository by using
the appropriate {\tt git} commands. You also need to submit an output file,
called {\tt output}, with at least one run of your program.

\vspace{-0.1in}
\subsection*{The {\tt while} Loop Overview}
\vspace*{-.05in}
The general form of a {\tt while} loop is:
\begin{Verbatim}[commandchars=\\\{\}]
          while (\textcolor{red}{\it condition} )
          \{
                \textcolor{red}{\it ... one or more Java statements ...}
          \}
          \end{Verbatim}
          \vspace{-0.15in}
          As long as the \textcolor{red}{\it condition} is true, the body of the loop
          will execute, and the loop will
          keep repeating. The condition is tested at the ``top'' of the loop.

          \noindent Here is an example that shows how to keep generating random throws of a pair of dice
          until a ``double'' (both dice are the same) is rolled:
          \vspace{-0.2in}
          \begin{verbatim}
          ...
        int d1 = -1, d2 = -2; // two dice (initialized to different values)
        int numTries = 0;
        while (d1 != d2) // keep generating random rolls until two are equal
        {
            d1 = rand.nextInt(6)+1;
            d2 = rand.nextInt(6)+1;
            numTries++;
        }
        System.out.println("After " + numTries + " rolls of the dice, two were equal");
          ...
          \end{verbatim}

          \vspace*{-.15in}
          \subsection*{General Guidelines for Practical Sessions}
          \vspace*{-.05in}
          \begin{itemize}
            \item {\bf Submit \textbf{\textit{Something}}.} Your grade for this assignment is a ``checkmark'' indicating whether you
              did or did not complete the work and submit something to the Bitbucket repository using the ``{\tt git add}'', ``{\tt
              git commit}'', and ``{\tt git push}'' commands.

            \item {\bf Update Your Repository Often!} You should {\tt add}, {\tt commit}, and {\tt push} your updated files each
              time you work on them, always including descriptive messages about each code change.

            \item {\bf Review the Honor Code Policy on the Syllabus.} Remember that while you may discuss your work with other
              students in the course, code that is nearly identical to, or merely variations on, the work of others will be
              taken as evidence of violating the \mbox{Honor Code}.

          \end{itemize}
          \end{document}

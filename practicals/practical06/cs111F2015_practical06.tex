% CS 111 style
% Typical usage (all UPPERCASE items are optional):
%       \input 111pre
%       \begin{document}
%       \MYTITLE{Title of document, e.g., Lab 1\\Due ...}
%       \MYHEADERS{short title}{other running head, e.g., due date}
%       \PURPOSE{Description of purpose}
%       \SUMMARY{Very short overview of assignment}
%       \DETAILS{Detailed description}
%         \SUBHEAD{if needed} ...
%         \SUBHEAD{if needed} ...
%          ...
%       \HANDIN{What to hand in and how}
%       \begin{checklist}
%       \item ...
%       \end{checklist}
% There is no need to include a "\documentstyle."
% However, there should be an "\end{document}."
%
%===========================================================
\documentclass[11pt,twoside,titlepage]{article}
%%NEED TO ADD epsf!!
\usepackage{threeparttop}
\usepackage{graphicx}
\usepackage{latexsym}
\usepackage{color}
\usepackage{listings}
\usepackage{fancyvrb}
%\usepackage{pgf,pgfarrows,pgfnodes,pgfautomata,pgfheaps,pgfshade}
\usepackage{tikz}
\usepackage[normalem]{ulem}
\tikzset{
    %Define standard arrow tip
%    >=stealth',
    %Define style for boxes
    oval/.style={
           rectangle,
           rounded corners,
           draw=black, very thick,
           text width=6.5em,
           minimum height=2em,
           text centered},
    % Define arrow style
    arr/.style={
           ->,
           thick,
           shorten <=2pt,
           shorten >=2pt,}
}
\usepackage[noend]{algorithmic}
\usepackage[noend]{algorithm}
\newcommand{\bfor}{{\bf for\ }}
\newcommand{\bthen}{{\bf then\ }}
\newcommand{\bwhile}{{\bf while\ }}
\newcommand{\btrue}{{\bf true\ }}
\newcommand{\bfalse}{{\bf false\ }}
\newcommand{\bto}{{\bf to\ }}
\newcommand{\bdo}{{\bf do\ }}
\newcommand{\bif}{{\bf if\ }}
\newcommand{\belse}{{\bf else\ }}
\newcommand{\band}{{\bf and\ }}
\newcommand{\breturn}{{\bf return\ }}
\newcommand{\mod}{{\rm mod}}
\renewcommand{\algorithmiccomment}[1]{$\rhd$ #1}
\newenvironment{checklist}{\par\noindent\hspace{-.25in}{\bf Checklist:}\renewcommand{\labelitemi}{$\Box$}%
\begin{itemize}}{\end{itemize}}
\pagestyle{threepartheadings}
\usepackage{url}
\usepackage{wrapfig}
% \usepackage{hyperref}
\usepackage[hidelinks]{hyperref}
%=========================
% One-inch margins everywhere
%=========================
\setlength{\topmargin}{0in}
\setlength{\textheight}{8.5in}
\setlength{\oddsidemargin}{0in}
\setlength{\evensidemargin}{0in}
\setlength{\textwidth}{6.5in}
%===============================
%===============================
% Macro for document title:
%===============================
\newcommand{\MYTITLE}[1]%
   {\begin{center}
     \begin{center}
     \bf
     CMPSC 111\\Introduction to Computer Science I\\
     Fall 2014\\
     \medskip
     \end{center}
     \bf
     #1
     \end{center}
}
%================================
% Macro for headings:
%================================
\newcommand{\MYHEADERS}[2]%
   {\lhead{#1}
    \rhead{#2}
    \immediate\write16{}
    \immediate\write16{DATE OF HANDOUT?}
    \read16 to \dateofhandout
    \lfoot{\sc Handed out on \dateofhandout}
    \immediate\write16{}
    \immediate\write16{HANDOUT NUMBER?}
    \read16 to\handoutnum
    \rfoot{Handout \handoutnum}
   }

%================================
% Macro for bold italic:
%================================
\newcommand{\bit}[1]{{\textit{\textbf{#1}}}}

%=========================
% Non-zero paragraph skips.
%=========================
\setlength{\parskip}{1ex}

%=========================
% Create various environments:
%=========================
\newcommand{\PURPOSE}{\par\noindent\hspace{-.25in}{\bf Purpose:\ }}
\newcommand{\SUMMARY}{\par\noindent\hspace{-.25in}{\bf Summary:\ }}
\newcommand{\DETAILS}{\par\noindent\hspace{-.25in}{\bf Details:\ }}
\newcommand{\HANDIN}{\par\noindent\hspace{-.25in}{\bf Hand in:\ }}
\newcommand{\SUBHEAD}[1]{\bigskip\par\noindent\hspace{-.1in}{\sc #1}\\}
%\newenvironment{CHECKLIST}{\begin{itemize}}{\end{itemize}}

\begin{document}
\MYTITLE{Practical 6\\16 October 2015\\Due in Bitbucket by midnight on 23 October 2015 \\ ``Checkmark'' grade}

\vspace*{-.2in}
\subsection*{Summary}
\vspace*{-.05in}

As a means of both practicing the extension of your own classes and methods and better understanding the structure of
different Java classes, you will study the given Java files and modify them by adding more functionality to certain
methods.  Additionally, you will use the Lightweight Java Visualizer (LJV) to automatically create diagrams, like those
seen in Chapter 4, that depict the state of Java objects.  Then, using the ``{\tt git add}'', ``{\tt git commit}'', and
``{\tt git push}'' commands you will upload your modified source code, object visualizations, and the output you obtain
from running the {\tt Practical6} program to your Git repository hosted by Bitbucket.

\vspace*{-.15in}
\subsection*{Review the Textbook}
\vspace*{-.05in}

Be sure to read Sections 4.1 through 4.4 of your textbook to learn more about writing your own classes, constructors, and
methods.  Please ensure that, as you read these sections, you study the technical diagrams of an object's state, as
found on pages 167 and 180 of the textbook. As you review this material, try to make a list of questions about concepts
that you do not yet fully understand.  In addition to discussing these questions with the teaching assistants and the
course instructor, please take your own steps to answering them as you complete this assignment.

\vspace*{-.15in}
\subsection*{Save and Study the Provided Classes}
\vspace*{-.05in}

\begin{sloppypar}
Using the ``{\tt git pull}'' command, download the files {\tt Octopus.java}, {\tt Utensil.java}, and {\tt Practical6.java}
from the course repository. Using the strategy that was described in a previous assignment, please copy these files into
your own Bitbucket repository in a directory called {\tt practical05/}. Study these programs first and make sure you
understand them. Can you better understand the structure and behavior of the methods provided by these classes by
drawing a technical diagram like the one in Figure 4.7 of your textbook? To compile them, you can type ``{\tt javac
  Practical6.java}'' in your terminal window; to run them, please type ``{\tt java Practical6}''.
\end{sloppypar}

\vspace*{-.15in}
\subsection*{Enhance the Classes}
\vspace*{-.05in}

\begin{enumerate}
\item

\begin{itemize}

% \item Edit the file {\tt Octopus.java} and find the constructor in this class. In the constructor, there is one
%   parameter {\tt n}, which contains a {\tt String}.  Change this by adding one more parameter, {\tt a}, of type {\tt
%     int}.  This is the age of the octopus. Save this in the appropriate instance variable (imitating what was done for
%     the name).

  \item Please find the constructor in the {\tt Octopus.java} file.  You will notice that, even though an
    instance of the {\tt Octopus} class has instance variables like {\tt weight} and {\tt name}, these variables are not
    initialized by the constructor. Assuming that the value of $-1$, indicating that the variable has not yet been set,
    is an acceptable starting value for both variables, please add to the constructor assignment statements initializing
    them to $-1$.

% \item \noindent Edit the file {\tt Practical6.java} and look for the place where variable {\tt ocky} is defined to be a
%   {\tt new Octopus}. Add an ``age'' to this so that we are specifying two things, not one, in the construction.  Delete
%   or comment out the ``{\tt ocky.setAge(10)}'' method call in the next line.

  \item You should also notice that the current constructor for {\tt Octopus} only accepts the ``{\tt String n}''
    parameter that allows you to specify the name of the object.  Since an {\tt Octopus} also has an {\tt age} and a
    {\tt weight}, you should write a new constructor that has three formal parameters---one for each of these instance
    variables---and initializes all of them correctly. You can use the ``{\tt Account}'' example in Section 4.4 as a
    source of inspiration for the way in which you add an improved constructor for the {\tt Octopus} class.

\end{itemize}

%\newpage

\item
\begin{itemize}

\item Now, use GVim to edit the file {\tt Practical6.java}. Declare a second {\tt Octopus} variable (don't just change
    the name of the one that's there---create another one) and assign it any name and age that you want. When you create
  this instance of the {\tt Octopus} class, you should use your newly defined constructor that takes multiple
  parameters.

\item \noindent Continuing to add code in the {\tt Practical6.java} file, please create a second {\tt Utensil} of any
  type you wish, imitating the declaration and initialization of {\tt spat}.  Assign a cost and a color to this utensil.
  Now, assign this utensil to the new {\tt Octopus} you created.

\item \noindent Print out the name, age, weight, and favorite utensil of your new {\tt Octopus} object. Finally, print
  out the type, cost, and color of your new utensil contained within the {\tt Octopus}.

\end{itemize}
\end{enumerate}

\vspace*{-.30in}
\subsection*{Visualize the Objects}
\vspace*{-.1in}

\begin{enumerate}
\item

\begin{itemize}

% \item Edit the file {\tt Octopus.java} and find the constructor in this class. In the constructor, there is one
%   parameter {\tt n}, which contains a {\tt String}.  Change this by adding one more parameter, {\tt a}, of type {\tt
%     int}.  This is the age of the octopus. Save this in the appropriate instance variable (imitating what was done for
%     the name).

  \item Please study the source code of {\tt Practical6.java} to find the location where it uses a static method to
    configure the Lightweight Java Visualizer class called {\tt LJV}. Now, find the method calls that produce a
    visualization of the {\tt ocky} instance of the {\tt Octopus} class and notice the name of the file that will
    contain the visualization. In your terminal, you can type ``{\tt evince ocky-before.pdf}'' to see what {\tt ocky}
    looks like in memory.

  \item After you have studied the ``before'' visualization, please again run the {\tt evince} program to view the file
    called ``{\tt ocky-after.pdf}''. What are the similarities and differences between these two diagrams? How did the
    method calls change the {\tt ocky} object?

\end{itemize}

\item
  \begin{itemize}

    \item In the previous phase of this assignment, you were responsible for adding new code to {\tt Practical6.java}
      that would construct another instance of the {\tt Octopus} class.  Please find this code and make sure that you
      understand how it works. To test your understanding, can you draw, on paper, what you think this object looks like
      in memory?

    \item Following the example of the previous calls to the {\tt LJV} class, please create a visualization of your
      new instance of the {\tt Octopus} class. Remember, you will not need to call the {\tt LJV} methods that configure
      the visualizer; instead, you only have to make a call to the {\tt drawGraph} method that will save the diagram in
      a file called ``{\tt my-ocky.pdf}''.

  \end{itemize}
\end{enumerate}

\vspace*{-.3in}
\subsection*{Completing the Practical Assignment}
\vspace*{-.1in}

To finish this assignment and earn a ``checkmark'', you should submit, through your Bitbucket repository, the {\tt
  Octopus.java} and {\tt Practical6.java} files you edited. You should also upload the three PDF files, as created by
running the ``{\tt java Practical6}'' program, in the repository.

\vspace*{-.15in}
\subsection*{General Guidelines for Practical Sessions}
\vspace*{-.05in}
\begin{itemize}

\item {\bf Submit \textbf{\textit{Something}}.} Your grade for this assignment is a ``checkmark'' indicating whether you
  did or did not complete the work and submit something to the Bitbucket repository. Please update your repository
  regularly and make sure to upload the final files on time.

% \item {\bf Update Your Repository Often!} You should {\tt add}, {\tt commit}, and {\tt push} your updated files each
%   time you work on them, always including descriptive messages about each code change.

\item {\bf Review the Honor Code Policy on the Syllabus.} Remember that while you may discuss your work with other
  students in the course, code that is nearly identical to, or merely variations on, the work of others will be
  taken as evidence of violating the \mbox{Honor Code}.
\end{itemize}


\end{document}

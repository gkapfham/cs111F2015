% CS 111 style
% Typical usage (all UPPERCASE items are optional):
%       \input 111pre
%       \begin{document}
%       \MYTITLE{Title of document, e.g., Lab 1\\Due ...}
%       \MYHEADERS{short title}{other running head, e.g., due date}
%       \PURPOSE{Description of purpose}
%       \SUMMARY{Very short overview of assignment}
%       \DETAILS{Detailed description}
%         \SUBHEAD{if needed} ...
%         \SUBHEAD{if needed} ...
%          ...
%       \HANDIN{What to hand in and how}
%       \begin{checklist}
%       \item ...
%       \end{checklist}
% There is no need to include a "\documentstyle."
% However, there should be an "\end{document}."
%
%===========================================================
\documentclass[11pt,twoside,titlepage]{article}
%%NEED TO ADD epsf!!
\usepackage{threeparttop}
\usepackage{graphicx}
\usepackage{latexsym}
\usepackage{color}
\usepackage{listings}
\usepackage{fancyvrb}
%\usepackage{pgf,pgfarrows,pgfnodes,pgfautomata,pgfheaps,pgfshade}
\usepackage{tikz}
\usepackage[normalem]{ulem}
\tikzset{
    %Define standard arrow tip
%    >=stealth',
    %Define style for boxes
    oval/.style={
           rectangle,
           rounded corners,
           draw=black, very thick,
           text width=6.5em,
           minimum height=2em,
           text centered},
    % Define arrow style
    arr/.style={
           ->,
           thick,
           shorten <=2pt,
           shorten >=2pt,}
}
\usepackage[noend]{algorithmic}
\usepackage[noend]{algorithm}
\newcommand{\bfor}{{\bf for\ }}
\newcommand{\bthen}{{\bf then\ }}
\newcommand{\bwhile}{{\bf while\ }}
\newcommand{\btrue}{{\bf true\ }}
\newcommand{\bfalse}{{\bf false\ }}
\newcommand{\bto}{{\bf to\ }}
\newcommand{\bdo}{{\bf do\ }}
\newcommand{\bif}{{\bf if\ }}
\newcommand{\belse}{{\bf else\ }}
\newcommand{\band}{{\bf and\ }}
\newcommand{\breturn}{{\bf return\ }}
\newcommand{\mod}{{\rm mod}}
\renewcommand{\algorithmiccomment}[1]{$\rhd$ #1}
\newenvironment{checklist}{\par\noindent\hspace{-.25in}{\bf Checklist:}\renewcommand{\labelitemi}{$\Box$}%
\begin{itemize}}{\end{itemize}}
\pagestyle{threepartheadings}
\usepackage{url}
\usepackage{wrapfig}
% \usepackage{hyperref}
\usepackage[hidelinks]{hyperref}
%=========================
% One-inch margins everywhere
%=========================
\setlength{\topmargin}{0in}
\setlength{\textheight}{8.5in}
\setlength{\oddsidemargin}{0in}
\setlength{\evensidemargin}{0in}
\setlength{\textwidth}{6.5in}
%===============================
%===============================
% Macro for document title:
%===============================
\newcommand{\MYTITLE}[1]%
   {\begin{center}
     \begin{center}
     \bf
     CMPSC 111\\Introduction to Computer Science I\\
     Fall 2014\\
     \medskip
     \end{center}
     \bf
     #1
     \end{center}
}
%================================
% Macro for headings:
%================================
\newcommand{\MYHEADERS}[2]%
   {\lhead{#1}
    \rhead{#2}
    \immediate\write16{}
    \immediate\write16{DATE OF HANDOUT?}
    \read16 to \dateofhandout
    \lfoot{\sc Handed out on \dateofhandout}
    \immediate\write16{}
    \immediate\write16{HANDOUT NUMBER?}
    \read16 to\handoutnum
    \rfoot{Handout \handoutnum}
   }

%================================
% Macro for bold italic:
%================================
\newcommand{\bit}[1]{{\textit{\textbf{#1}}}}

%=========================
% Non-zero paragraph skips.
%=========================
\setlength{\parskip}{1ex}

%=========================
% Create various environments:
%=========================
\newcommand{\PURPOSE}{\par\noindent\hspace{-.25in}{\bf Purpose:\ }}
\newcommand{\SUMMARY}{\par\noindent\hspace{-.25in}{\bf Summary:\ }}
\newcommand{\DETAILS}{\par\noindent\hspace{-.25in}{\bf Details:\ }}
\newcommand{\HANDIN}{\par\noindent\hspace{-.25in}{\bf Hand in:\ }}
\newcommand{\SUBHEAD}[1]{\bigskip\par\noindent\hspace{-.1in}{\sc #1}\\}
%\newenvironment{CHECKLIST}{\begin{itemize}}{\end{itemize}}

\begin{document}
\MYTITLE{Practical 3\\18 September 2015\\Due in Bitbucket by midnight of the day of your practical \\ ``Checkmark'' grade}

\subsection*{Summary}

To practice using interactive programs implemented in the Java programming language. Additionally, to learn more about
how to declare variables of different data types and then to change values of these variables using expressions.  To
also continue to explore how to display graphics using the applet environment provided by Java. Finally, to continue
practicing the use of the {\tt git add}, {\tt git commit}, and {\tt git push} commands to upload Java source code and
graphics to your Git repository.

\vspace*{-.1in}
\subsection*{Review the Textbook}

To successfully complete this practical assignment, you should review Section 2.2's content that explains how to declare
variables and to assign values to these variables. Since the source code that is provided to you as part of this
assignment also solicits input from the user, students should also re-read Section 2.6's exposition of the methods
provided by the {\tt java.util.Scanner} class. Finally, study Sections 2.7 and 2.8 and their explanations of how to
create static graphics in the Java programming language.  Please see the instructor if you have questions about
these topics.

\vspace*{-.1in}
\subsection*{Preparing for the Programming Task}

Before you start enhancing the Java classes required by this assignment, you should separately type the commands ``{\tt cd
  cs111F2015}'', ``{\tt cd cs111F2015-<your user name>}'', and ``{\tt cd practicals}'' in your terminal window. Once you
are in the {\tt practicals/} directory of your Git repository, you can type the command ``{\tt mkdir practical03}'' to
create a new directory for this assignment. You can run the {\tt gvim} command from this directory when you are ready to
begin to implement the required Java program. Please see the instructor if you have problems with these preparatory steps.

Now, in the course's ``share'' repository, after you type ``{\tt git pull}'' command, go to the {\tt practical03/}
directory, where you will find two Java programs: {\tt DisplayDrawingCanvas.java} and {\tt PaintDrawingCanvas.java}.
Using the graphical file browser or the terminal, copy the {\tt practical03/} directory from the ``share''
repository to your own {\tt cs111F2015-<your user name>} repository inside the {\tt practicals/} directory. Before you
move on to the next steps, make sure that your repository is clearly organized with separate directories for labs and
practicals and \mbox{each assignment}.

\vspace*{-.1in}
\subsection*{Reading Input from the User}

Please use the {\tt gvim} text editor to carefully study the code in the {\tt DisplayDrawingCanvas.java} file. What are
the lines in this program that read input from the user and store the input values in variables? How do these lines of
source code work? Next, find the line of code that creates {\tt JFrame} and modify it so that it correctly prints your
name and the date. This is the only change that you need to make to this file! Please see the instructor if you do not
understand the code in this file.

\vspace*{-.1in}
\subsection*{Painting with Complementary Colors}

In this phase of the assignment, you will need to load the file {\tt PaintDrawingCanvas.java} file into {\tt gvim} so
that you can understand the existing code in this file and then add your own enhancements to this class. It is important
to note that this code refers to a variable declared in the {\tt DisplayDrawingCanvas} class through notation such as
{\tt DisplayDrawingCanvas.redValue}.

The purpose of this program is to create a canvas that contains two colors in it. The first colored rectangle---which
should take up exactly half of the page---should display the color that was requested by the user. For instance, if the
user inputs the RGB value of $(255,0,0)$, then this region of the graphic should be filled with the color red. The
second half of the image should be filled with the complement of the color requested by the user. For example, the
complement of the red value of $255$ is the value $255-255=0$ and the complement of the green value of $0$ is
$255-0=255$. Intuitively, taking the complement of an RGB should give you the color that is on the opposite side of the
color wheel that you can view in the {\tt gimp} program. So, what is the RGB value of the color red? Does this color
appear on the opposite side of the color wheel in {\tt gimp}?

As a means of practicing the use of variables and expressions, and the creation of graphics in Java, you should make
some small additions to this program so that it fulfills its intended purpose. To accomplish this task, the first thing
that you must do is add a call to {\tt page.fillRect} method so that it paints a rectangle of the color requested by the
user. Next, you should store in the variable called {\tt userComplementaryColor} a color that is the complement of the
one requested by the user. This means that you will have to write in the form of a Java expression the simple equations
that were intuitively explained in the previous paragraph.

\subsection*{Using Version Control Correctly}

As you are typing your program in the {\tt gvim} text editor, you should regularly save your files.  Once you have
created a preliminary version of your program and it compiles and runs as anticipated, you should use the ``{\tt git
add}'' command to stage it in your Git repository.  Next, you can use the ``{\tt git commit}'' command to save it in
your local repository with a version control message.  Finally, you can run ``{\tt git push}'' to transfer your file to
the Bitbucket servers.  For this practical assignment, you do {\em not} have to hand in a hard copy of anything---just
upload your Java program and an output produced by your program to Bitbucket using the appropriate {\tt git} commands.
Please review your ``Git Cheatsheet'' and talk with a member of the class, the course instructor, or a teaching
assistant if you do not understand how to correctly use the Git version control system.






\vspace*{-.2in}
\subsection*{General Guidelines for Practical Sessions}

Since this is one of our first practical assignments and you are still learning how to use the Java programming
language, don't become frustrated if you make a mistake. Instead, use your mistakes as an opportunity for learning both
about the necessary technology and the background and expertise of the other students in the class, the teaching
assistants, and the instructor.

\vspace*{-.05in}
\begin{itemize}

% \itemsep -0.01in
\itemsep 0in

\item {\bf Experiment!} Practical sessions are for learning by doing without the pressure of grades or ``right/wrong''
  answers. So try things!  The best way to learn is by intelligently experimenting.

\item {\bf Submit \textbf{\textit{Something}}.} Your grade for this assignment is a ``checkmark'' indicating whether you
  did or did not complete the work and submit something to the Bitbucket repository using the ``{\tt git add}'', ``{\tt
    git commit}'', and ``{\tt git push}'' commands.

\item {\bf Practice Key Laboratory Skills.} As you are completing this assignment, practice using the {\tt gvim} text
  editor and the Ubuntu terminal until you can easily use their most important features.  Additionally, ask
  a teaching assistant or the course instructor to teach you some of the advanced features of {\tt gvim} and the
  terminal, thereby helping you to work more effectively.

\item {\bf Try to Finish During the Class Session.} Practical exercises are not intended to be the equal of the
  laboratory assignments. If you are simply a slow typist, I've given you until the end of the day, but ideally you
  should upload a file, even a partially working one, by the end of the class period. So, make sure that you correctly
  upload your file to your Git repository!

  % You also should ensure that, for this assignment and all subsequent assignments, you can confidently upload files to
  % your Git repository during the practical session.

\item {\bf Help One Another!} If your neighbor is struggling and you know what to do, offer your help. Don't ``do the
  work'' for them, but advise them on what to type or how to handle things. If you are stuck on a part of this practical
  session and you could not find any insights in either your textbook or online sources, formulate an intelligent
  question to ask your neighbor, a teaching assistant, or a course instructor. Try to strike the right balance between
  asking for help when you cannot solve a problem and working independently to find a solution.

\item {\bf Update Your Repository Often!} You should {\tt add}, {\tt commit}, and {\tt push} your updated files each
  time you work on them, always including descriptive messages about each code change.

\item {\bf Review the Honor Code Policy on the Syllabus.} Remember that while you may discuss programs with other
  students in the course, programs that are nearly identical to, or merely variations on, the work of others will be
  taken as evidence of violating the Honor Code.

\end{itemize}

\end{document}
